%%%%%%%%%%%%%%%%%%%%%%%%%%%%%%%%%%%%%%%%%
\documentclass[10pt,a4paper,sans]{moderncv} %% Font sizes: 10, 11, or 12; paper sizes: a4paper, letterpaper, a5paper,
%%legalpaper, executivepaper or landscape; font families: sans or roman
\usepackage[utf8]{inputenc} %% codificación para la inserción acentos y otros carácteres especiales
\usepackage[T1]{fontenc} %% codificación para la impresión de acentos y otros carácteres especiales
\usepackage[english]{babel}
\graphicspath{{../images/}}
\moderncvstyle{casual} %% CV theme - options include: 'casual' (default), 'classic', 'oldstyle' and 'banking'
\moderncvcolor{blue} %% CV color - options include: 'blue' (default), 'orange', 'green', 'red', 'purple', 'grey' and 'black'

%% \usepackage{lipsum} %% Used for inserting dummy 'Lorem ipsum' text into the template
\usepackage{multicol}
\AfterPreamble{\hypersetup{colorlinks=true}}
\usepackage[scale=0.75]{geometry} % Reduce document margins
%\setlength{\hintscolumnwidth}{3cm} % Uncomment to change the width of the dates column
%\setlength{\makecvtitlenamewidth}{10cm} % For the 'classic' style, uncomment to adjust the width of the space allocated to your name

%----------------------------------------------------------------------------------------
%	NAME AND CONTACT INFORMATION SECTION
%----------------------------------------------------------------------------------------

\firstname{Albert M.} % Your first name
\familyname{Orozco Camacho} % Your last name
% All information in this block is optional, comment out any lines you don't need
\address{356-404 Río Mixcoac Ave.}{03240 Mexico City, Mexico}
%% \mobile{(+52 1) 55 3262 1338}
\email{alorozco53@ciencias.unam.mx}


%----------------------------------------------------------------------------------------

\begin{document}

\makecvtitle % Print the CV title

\noindent
\textit{
  I have a broad spectrum of interests in AI: from the theoretical foundations of deep learning to applications regarding\
  NLP, chatbots, and social networks. My current goals are directed towards enhancing how humans and machines\
  communicate and understand themselves, as well as providing elegant models for such tasks.
}

\section{Personal Details}
%% \cvitem{Date of birth}{July 16th, 1994}
\cvitem{Website}{\url{https://alorozco53.github.io/}}
\cvitem{GitHub page}{\url{https://github.com/alorozco53}}
\cvitem{Linkedin page}{\url{https://www.linkedin.com/in/alorozco53/}}
%% \cvitem{Permanent address}{356-404 Río Mixcoac Ave. 03240 Mexico City, Mexico}
%% \cvitem{Mobile No.}{(+52 1) 55 3262 1338}

\section{Areas of Interest}
\cvlistitem{Computer Science}
\cvlistitem{Artificial Intelligence}
\cvlistitem{Machine Learning}

%----------------------------------------------------------------------------------------
%	EDUCATION SECTION
%----------------------------------------------------------------------------------------

\section{Education}

\cventry{2012-2017}{Bachelor in Science Degree}{
  Facultad de Ciencias, Universidad Nacional Aut\'{o}noma de M\'{e}xico (UNAM)}{Mexico City}{\textit{GPA -- 8.82 / 10}
}{Computer Science}
\cventry{2015}{Attended the 2015 \href{https://www2.ee.washington.edu/news/2015JelinekWorkshopSummerSchool.html}{Jelinek Summer School on Human Language Technologies}}
        {Summer school}
        {It was a 2-week introductory summer school on cutting-edge topics about speech recognition, machine learning, and natural language processing}{It was held at the University of Washington, Seattle}{Received a scholarship from The North American Chapter of the Association for Computational Linguistics (NAACL) to attend the summer school.}
%% \cventry{2009--2012}{High School}{
%%   Prepa Tec de Monterrey, Campus Guadalajara}{Guadalajara, Jalisco, Mexico}{\textit{GPA -- 91 / 100}
%% }{Secondary School Certificate}

\section{Theses (Final Projects)}

\cventry{BSc.}
        {Automatic Generation of Internet Memes using a Deep Neural Network}
        {Supervised by \href{http://turing.iimas.unam.mx/~ivanvladimir/}{Dr Ivan V. Meza}}
        {}{}
        {\href{http://132.248.9.195/ptd2018/febrero/0770173/}{Faculty of Sciences, UNAM.} pp. 111.}

\section{Publications}

\cventry{2016}
        {LIPN-IIMAS at SemEval-2016 Task 1: Random Forest Regression Experiments on Align-and-Differentiate and Word Embeddings penalizing strategies}
        {}{}{}
        {Lightgow, O. , Meza, I. , Orozco, A. , Garcia-Flores, J. y Buscaldi, D.
         \href{https://www.aclweb.org/anthology/S/S16/S16-1112.pdf}{Proceedings of the 10th International Workshop on Semantic Evaluation (SemEval-2016)} pp. 726-731.}

\cventry{2014}
        {The Golem Team, RoboCup@Home 2014}{Technical Report}
        {}{}
        {Pineda, L. , Rascon, C. , Fuentes, G. , Estrada, V. , Rodriguez, A. , Meza, I. , Ortega, H. ,
         Reyes, M. , Peña, M. , Duran, J. , Campos, E. , Chimal, S. y Orozco, A.
        \href{http://golem.iimas.unam.mx/pubs/tdp_Golem-II+_2014.pdf}{DCC, IIMAS, UNAM} pp. 8.}

\section{Workshops}

\cventry{2018}
        {Poster to present: "Challenges in Automatic Meme Generation Using Deep Neural Networks"}
        {Conference presentation}
        {The poster is related to what I've been doing to my bachelor's thesis project lately}
        {It will be presented at \href{http://www.latinxinai.org}{LatinX in AI} workshop,
          held at \href{https://nips.cc}{NIPS} in Montréal, Canada}
        {}

\cventry{2018}
        {Poster to present: "Challenges in Automatic Meme Generation Using Deep Neural Networks"}
        {Conference presentation}
        {The same poster as for LatinX in AI 2018 workshop}
        {It will be part of the third Self-Organizing Conference on Machine Learning (\href{https://sites.google.com/view/socml-2018/home}{SOCML}),
          held at the Google offices in Toronto, Canada; it is organized by Ian Goodfellow}
        {}

\cventry{2017}
        {Poster presented: "Deep Automatic Internet Meme Caption Generation"}
        {Conference presentation}
        {I presented a poster at the second SOCML event, held at Sunnyvale, USA}
        {The poster can be visualized \href{https://drive.google.com/file/d/1DPjqiXcmliwPZBOfBIXxnIX_rs9EPUub/view?usp=sharing}{here}}
        {}

\cventry{2017}
        {Presented a talk titled "Towards Automatic Meme Generation"}
        {Outreach talk}
        {I spoke about deep learning applications to NLP, from a linguistic perspective}
        {\emph{CoLiCo} stands for ``Computational Linguistics Colloquium'' and was organized by UNAM Linguistic Engineering Group
          Speaker at \href{http://www.corpus.unam.mx/colico/VIIICoLiCo.html}{CoLiCo} held at UNAM Facultad de Filosofía y Letras}
        {The slides are available in \url{https://alorozco53.github.io/talks/onto_memes.html}}


\section{Talks}

\cventry{2017}
        {Presented a talk titled "Towards Automatic Meme Caption Generation"}
        {Outreach talk}
        {I spoke about my undergraduate thesis project and the lessons I learned during the coding part}
        {Invited speaker at \href{https://www.meetup.com/GDG-UNAM/}{UNAM Google Developer Group}'s meetup and \href{http://www.escom.ipn.mx}{IPN ESCOM}}
        {The slides are available in \url{https://alorozco53.github.io/talks/lessons.html}}

\cventry{2017}
        {Presented a talk titled "Implementing Eyes on your Chatbot"}
        {Outreach talk}
        {Invited speaker at the "Bots LATAM" community, whose goal is to gather the most enthusiastic people in AI and Chatbots together in Mexico City}
        {}
        {The slides are available in \url{https://alorozco53.github.io/talks/eyes_on_bot.html}}


\section{Internships and Professional Experience}

\cventry{2017-}{Data Scientist at \href{http://www.mariachi.io}{Mariachi IO}}
        {Paid Job}
        {I joined \emph{Mariachi IO} to help out in the solution of several tasks that require NLP, machine learning, and image processing}
        {I am currently building a computer vision application using classic OpenCV-based algorithms and state-of-the art deep learning tools}
        {}
        {}

\cventry{2017-}{Data Science Consultant at \href{http://www.fractalabogados.com}{Fractal Abogados}}
        {Paid Job}
        {\emph{Fractal Abogados} is a startup whose purpose is to provide feasible IT (and AI) solutions into today's Mexican (and Latin American) law system}
        {I work as an AI consultant in Fractal's signature project: a legal chatbot. \href{https://m.me/fractal-abogados}{\emph{Max}} is a Facebook Messenger based virtual assistant, powered by IBM Watson that automates the most common legal advices in Mexico}
        {}
        {}

\cventry{2016-}{Teacher assistant the UNAM's Facultad de Ciencias}
        {Paid Job}
        {Taught \href{http://turing.iimas.unam.mx/~ivanvladimir/page/curso_rpyaa}{Machine Learning and Pattern Recognition} and \href{https://sites.google.com/site/automataslengformales20172/}{Automata and Formal Languages} during 2017. Taught the \href{https://sites.google.com/site/lengprog20162/}{Programming Languages} during the 2016 Spring Semester. Taught \href{https://sites.google.com/a/ciencias.unam.mx/estructuras-discretas/home}{Discrete Structures Lab} and \href{https://sites.google.com/site/logcompunam20171/home}{Computational Logic} during the 2016 Fall Semester. All courses are offered for the undergraduate curriculum}
        {I will be teaching \emph{Computational Logic} again this Spring 2018 semester}{}{}
\cventry{2013-2015}{Student / researcher at UNAM's \textit{Grupo Golem}}
        {Extra-curricular activity}
        {Grupo Golem is a research group at IIMAS (Instituto de Investigaciones en Matem\'{a}ticas Aplicadas y en Sistemas) whose main goal is to model the cognitive interaction between a humans and computer; all the research is unified in a service robot that competes internationally in the RoboCup@Home competition; the group's website is \href{http://golem.iimas.unam.mx/home.php?lang=en&sec=home}{this one}}
        {Grupo Golem's leader is \href{http://turing.iimas.unam.mx/~luis/}{Dr Luis A. Pineda}}{}{}
\cventry{2012-2013}{Student / researcher at UNAM's IIMAS's Computer Science Department}
        {Extra-curricular activity}
        {I work alongside \href{http://turing.iimas.unam.mx/~ivanvladimir/}{Dr Ivan V. Meza} with speeech recognizers, and replicated an experiment in which a robot learns from a human teacher how to transform babblings to simple English words}
        {}{}{}


\section{Other Activites}

\cventry{2017}
        {Attended a NLP hackathon (\emph{Gilkatón}) organized by UNAM's
          \href{http://grupos.iingen.unam.mx/iling/es-mx/Paginas/default.aspx}{Grupo de Ingeniería Lingüística}}
        {Hackathon}
        {Held in Mexico City at the Engineering Tower, UNAM}
        {Alongside two linguists, we created a program to extract relevant information
          from a corpus of legal documents}
        {The code developed can be found here: \url{https://github.com/alorozco53/Gilkaton}.}

\cventry{2017}
        {Attended an AI-Chatbot Hackathon organized by \href{http://synx.co}{Synx} and \href{https://www.recime.io}{Recime}}
        {Hackathon}
        {Held in Tlaquepaque, Jalisco, Mexico at \href{http://www.iteso.mx}{ITESO}}{}
        {Alongside three friends, I helped developing a NLP module for a chatbot that tracks the user's food quality and exercising activites.}

\cventry{2017}
        {Volunteer in \href{http://technovationmx.org}{Technovation Challenge}}{Volunteering}
        {This event's focus is to encourage programming and entrepreneurship in Secondary School girls. During one of the regional knockout events, I helped out with logistic issues; it was held during May at Mexico City}
        {}{}

\cventry{2016}
        {Participated in the second annual ``Jakatón'' (computational linguistics hackathon)}{Hackathon}
        {Held in Cholula, Puebla, Mexico}{}{}

\cventry{2015}{Participated in \href{http://naacl.org/reports/emerging_regions/Jakaton_report_apr2015.pdf}{Mexico's first computational linguistics hackathon}}{Hackathon}{I teamed up with a friend of mine to be finalists and obtained an honor's award}{}{}


%----------------------------------------------------------------------------------------
%	LANGUAGES SECTION
%----------------------------------------------------------------------------------------

\section{Languages}

\cvitemwithcomment{Spanish}{Mothertongue}{}
\cvitemwithcomment{English}{Bilingual proficiency}{Completely fluent (IELTS)}
\cvitemwithcomment{French}{Full professional proficiency}{Fluency mostly written (DELF B1)}

%----------------------------------------------------------------------------------------
%	COMPUTER SKILLS SECTION
%----------------------------------------------------------------------------------------

\section{Software}

\renewcommand{\listitemsymbol}{-~}
\subsection{Programming Languages}
\cvlistdoubleitem{Python}{Java}
\cvlistdoubleitem{C}{Haskell}
\cvlistdoubleitem{C++}{Prolog}
%% \cvlistdoubleitem{C++}{NumPy}
%% \cvlistdoubleitem{SciPy}{R}
%% \cvlistdoubleitem{Sci-kit Learn}{spaCy}
%% \cvlistdoubleitem{NLTK}{Matplotlib}
%% \cvlistdoubleitem{OpenCV}{PostgreSQL}

\subsection{Additional Software}
\cvlistdoubleitem{Tensorflow}{Keras}
\cvlistdoubleitem{SciPy}{Pandas}
\cvlistdoubleitem{PyTorch}{spaCy}
\cvlistdoubleitem{NumPy}{Scikit-Learn}
\cvlistdoubleitem{NLTK}{OpenCV}
%% \cvlistdoubleitem{Keras}{Bash (Shell)}
%% \cvlistdoubleitem{Amazon AWS}{Google Cloud}

%% \section{Personal Skills}
%% \cvlistitem{Ability to tackle complex problems.}
%% \cvlistitem{Ability to abstract the most important features of a given task.}
%% \cvlistitem{Critical thinking.}
%% \cvlistitem{Ability to work under pressure}
%% \cvlistitem{Provide efficient programming solutions to any problem.}

%----------------------------------------------------------------------------------------
%	INTERESTS SECTION
%----------------------------------------------------------------------------------------

%% \section{Hobbies}

%% \renewcommand{\listitemsymbol}{-~} % Changes the symbol used for lists

%% \cvlistdoubleitem{Science-fiction}{Robotics}
%% \cvlistdoubleitem{Soccer}{American football}
%% \cvlistdoubleitem{Science (in general)}{Exercising}

\end{document}
